%#!lualatex
\documentclass[lualatex,a4paper,11pt]{jlreq}
\usepackage{luatexja-fontspec}
\newjfontfamily{\HaraMC}[
    FontFace = {l}{n}{*-Light},
    FontFace = {m}{n}{*-Regular},
    FontFace = {b}{n}{*-Bold},
    BoldFont = {*-Bold},
  ]{HaranoAjiMincho}
\newjfontfamily{\HaraGT}[
    FontFace = {m}{n}{*-Regular},
    FontFace = {b}{n}{*-Bold},
    FontFace = {eb}{n}{*-Heavy},
    BoldFont = {*-Bold},
  ]{HaranoAjiGothic}
\newjfontfamily{\HaraMG}[
    FontFace = {m}{n}{*-Medium},
  ]{HaranoAjiGothic}

%\usepackage{luatexja-ruby}% for kenten
% [TODO] https://ja.osdn.net/projects/luatex-ja/ticket/41023
% => based on okumacro.sty
\makeatletter
\def\kenten#1{%
  \setbox1=\hbox to\z@{\hbox to 1\zw{\hss ・\hss}\hss}%
  \ht1=.63\zw
  \@kenten#1\end\relax}
\def\@kenten#1#2{%
  \ifx#1\end \let\next=\relax \else
    \raise.75\zw\copy1\nobreak #1\ifx#2\end\else
    \hskip\ltjgetparameter{kanjiskip}\relax\fi
    \let\next=\@kenten
  \fi\next#2}
\makeatother

%%%%%%%%%%%%%%%%
% additional packages
\usepackage{amsmath}
\usepackage{array}
\usepackage{bm}
% \usepackage[defaultsups]{newpxtext}
% \usepackage[zerostyle=c,straightquotes]{newtxtt}
% \usepackage{newpxmath}
\usepackage[svgnames]{xcolor}
\usepackage[hyperfootnotes=false]{hyperref}
\usepackage{hologo}
\usepackage{xspace}

% フォントエラー対策のため追加
\usepackage{fontspec}
\usepackage{unicode-math}

\setmainfont{TeX Gyre Pagella} 
\setsansfont{TeX Gyre Heros} % サンセリフの例
\setmonofont{TeX Gyre Cursor} % 等幅フォントの例
\setmathfont{STIX Two Math}

\iffalse
\def\Pkg#1{\textsf{#1}}
\def\Opt#1{\texttt{#1}}
\def\ClsY#1{\Pkg{\textcolor{Black}{#1}}}
\def\ClsT#1{\Pkg{\textcolor{Green}{#1}}}
\def\ClsYT#1{\Pkg{\textcolor{Blue}{#1}}}
\def\file#1{\textsf{#1}}
\def\code#1{\texttt{#1}}
\def\cs#1{\texttt{\textbackslash#1}}
\fi

\usepackage[japanese,english]{babel}% References in English
\AtBeginDocument{\def\figurename{Figure~}}% add a space

\usepackage{listings}
\lstset{%
  language=[LaTeX]{TeX},
  basicstyle={\small\ttfamily},
  texcsstyle={*\color{blue}},
  commentstyle={\color{Green}},
  columns=[l]{fullflexible},
  numbers=left,
  xrightmargin=0\zw,
  xleftmargin=3\zw,
  aboveskip=3pt,
  belowskip=3pt,
  moretexcs={setlength,setmainfont,maketitle,RequirePackage}
}

%\usepackage{shortvrb}
%\MakeShortVerb*{|}
%%%%%%%%%%%%%%%%

% foreign text -> italic
%\def\Foreign#1{\textit{#1}}



%def\_{\leavevmode\vrule width .45em height -.2ex depth .3ex\relax}

\frenchspacing
\begin{document}

\title{\textsf{\textbf{縮小を用いた実データにおけるポートフォリオ構築の最適化について}}}
\author{國廣 壱生
\thanks{\url{https://texjp.org},\ e-mail: \texttt{issue(at)texjp.org}}}
\maketitle


\hfil {\textbf 要約}
\vskip1em

\begin{center}
\begin{minipage}{0.8\textwidth}
  現代のポートフォリオ理論は,Markowitz(1952)によって確立された平均・分散アプローチに基づいており,所与の期待リターンに対してリスクを最小化することを目的とする.
  この最適化計算の核心には,資産リターンの標本共分散行列の推定が必要となる.
  しかし,ポートフォリオ構築時に通常用いられる標本共分散行列は,銘柄数が多い現実の状況下で推定誤差が大きいため,投資比率が不安定化する.
  この問題を解決するため,縮小という手法を用いて標本共分散行列の構造を変化させ,推定誤差を小さくする.
  これは比較的新しい手法で,実データにおける検証が不十分である.
  そのため本研究では,過去の株価データを用いて推定した共分散行列に縮小を適用し,従来よりも安定かつ効率的なポートフォリオ構築が行えるかを評価する.
\end{minipage}
\end{center}
\vskip2em

\section{導入}
本研究では,ポートフォリオ最適化問題における共分散行列の推定精度向上を目的とし,Ledoit-Wolfの縮小推定法を日本の株式市場データに適用した結果について述べる.特に,標本共分散行列が抱える推定誤差の問題と,それを縮小推定によってどのように改善できるかに焦点を当てる.

\section{関連手法}

\subsection{ポートフォリオ最適化問題}
% 研究の核である「ポートフォリオ最適化問題」に触れつつ,一番手法として古い標本共分散行列を用いた最適化手法を話す.
% ここで,標本共分散行列では推定誤差が大きいという問題があると指摘して,新しい手法が必要な理由を説明

\subsubsection{最小分散ポートフォリオ}

現代ポートフォリオ理論は,Harry Markowitzが1952年に発表した論文で提唱した最適化理論を元に構築されており,投資家が直面するリターンとリスクのトレードオフを定量的に解決する方法に従う.
この理論の目的は特定のリスク水準の下で期待リターンを最大化し,特定の期待リターン水準を達成する中でリスクを最小化することにある.

ポートフォリオの期待リターン $\mathbb{E}(r_p)$ は,個々の資産の期待リターン $\mathbb{E}(r_i)$ を,その資産の投資ウェイト $w_i$ で加重平均した値で求める.

\begin{align}
    \mathbb{E}(r_p) = \sum_{i=1}^{N} w_i \mathbb{E}(r_i) = \symbf{w}^\top \symbf{\mu}
\end{align}

$\symbf{w}$ は各資産に対する投資比率を縦に並べたウェイトベクトルであり, $\symbf{\mu}$ は各資産の期待リターンを並べた列ベクトルである.
また,ポートフォリオの分散 $\sigma^2_p$ は,ポートフォリオのリスクの大きさを表す.単に各資産の分散を合計するのではなく,資産間の共分散を計算することで求める.

\begin{align}
  \sigma^2_p = \sum_{i=1}^{N} \sum_{j=1}^{N} w_i w_j \text{Cov}(r_i, r_j) = \symbf{w}^\top \symbf{\Sigma} \symbf{w}
\end{align}

$\text{Cov}(r_i, r_j)$ は資産 $i$ のリターン $r_i$ と資産 $j$ のリターン $r_j$ の共分散でリターンの相関を示し, $\symbf{\Sigma}$ はすべての資産ペアの共分散と分散をまとめた分散共分散行列である.
この分散の計算において,資産間の共分散が小さい (または負の相関がある) ほど,ポートフォリオ全体のリスクは小さくなる.
そのため,ポートフォリオを最適化するには,分散共分散行列である $\symbf{\Sigma}$ を最小化する必要がある.

したがって,最小分散ポートフォリオを求める最小化問題は以下のように定式化できる.
\begin{align}
  \min_{\symbf{w}} \quad \symbf{w}^\top \symbf{\Sigma} \symbf{w}
  \\[1em]
  \text{s.t.} \quad \symbf{C}^\top\symbf{w} = \symbf{\gamma}
\end{align}

$\symbf{C}$ は, $N$ を資産数, $M$ を制約数としたときにおける $N \times M$ の行列であり,$\symbf{\gamma}$ は各制約条件の目標値を格納した $M \times 1$ の列ベクトルである.
この最小化問題を解くことによって,リスクを最小限に抑える最適なウェイトベクトル $\symbf{w}$ が以下のように求められることが知られている.
\begin{align}
  \symbf{w} &= \symbf{\Sigma}^{-1} \symbf{C} (\symbf{C}^\top \symbf{\Sigma}^{-1} \symbf{C})^{-1} \symbf{\gamma} \\ \notag
             &= \sum_{i=1}^{N} \frac{\symbf{C}^\top \cdot \symbf{u}_i}{\lambda_i} \symbf{u}_i
\end{align}
ここで,$\lambda_i$は分散共分散行列の各固有値を表し,$\symbf{u}_i$は分散共分散行列の各固有ベクトルを表す.

\subsubsection{標本共分散行列}

理論的なポートフォリオ最小化問題では,真の分散共分散行列 $\symbf{\Sigma}$ を用いて推定されるが,実務上では $\symbf{\Sigma}$ を観測することは不可能であるため,
過去の収益率 $\symbf{R}$ から推定した標本共分散行列 $\symbf{\hat{\Sigma}} = \frac{1}{T-1} \symbf{R}^\top \symbf{R}$ が用いられる.

実務上では,ポートフォリオの最適化は(3)式を用いて以下のように行われる.
\begin{align}
  \min_{\symbf{w}} \quad \symbf{w}^\top \hat{\symbf{\Sigma}} \symbf{w}
  \\[1em]
  \text{s.t.} \quad \symbf{C}^\top\symbf{w} = \symbf{\gamma}
\end{align}

また,(5)式も標本共分散行列を用いて以下のように表される.
\begin{align}
  \symbf{w} &= \hat{\symbf{\Sigma}}^{-1} \symbf{C} (\symbf{C}^\top \hat{\symbf{\Sigma}}^{-1} \symbf{C})^{-1} \symbf{\gamma} \\ \notag
             &= \sum_{i=1}^{N} \frac{\symbf{C}^\top \cdot \symbf{u}_i}{\lambda_i} \symbf{u}_i
\end{align}

しかし,特に銘柄数 $N$ が観測期間 $T$ に対して大きい場合,標本共分散行列は以下のような問題を引き起こすことが知られている.

第一に,次元性の呪いと推定誤差である.推定すべきパラメータ数 $N(N+1)/2$ がデータ量 $N \times T$ に対して過大である場合,標本共分散行列は真の行列に対する一致推定量とならず,大きな推定誤差を含むことになる.

% しかし,銘柄数 $N$ が観測数 $T$ よりも多い場合では,分散共分散行列の推定に必要な観測数 $T$ が少ないため,取得できるデータが希薄になり標本誤差が大きくなってしまう.
% このとき,標本共分散行列のランクは $T$ 以下となり,本来の行列の次元 $N$ よりも小さくなってしまうため逆行列を持たなくなる.
% また,銘柄数 $N$ が観測数 $T$ に近い場合でも,標本共分散行列が持つ最小固有値の値が限りなくゼロの方向へ歪んで推定される.そのため,ポートフォリオの最適なウェイトベクトルを計算する際にほぼ無限大のウェイトが割り当てられてしまう.
% これは,ポートフォリオの真のリスクを著しく過小評価することになり,正しくポートフォリオの最適なウェイトベクトルが推定できない.
% したがって,標本共分散行列をポートフォリオ最適化問題に使うことは不適切であり,他の手法を用いてポートフォリオ最適化問題を解く必要がある.

\subsection{高次元共分散行列の推定手法}

本研究では,標本共分散行列の問題点を克服するために提案されたいくつかの手法を比較検討する.

\subsubsection{市場ファクターモデル (Market Factor Model)}
市場全体の動きを表す単一のファクター(例えばTOPIXなどの市場インデックス)を用いて個別の資産リターンを説明するモデルである.
資産 $i$ のリターン $r_i$ は,市場ファクター $f$ とその感応度 $\beta_i$,および固有リスク $\epsilon_i$ を用いて以下のようにモデル化される.
\begin{align}
    r_i = \beta_i f + \epsilon_i
\end{align}
このモデルを用いることで,共分散行列の推定パラメータ数を大幅に削減し,推定誤差を抑制することができる.

\subsubsection{主成分分析 (PCA) と POET}
PCA(主成分分析)を用いてデータから主要な共通成分(ファクター)を統計的に抽出し,共分散行列を構築する手法である.
事前に定義された市場インデックスを用いるのではなく,データそのものから支配的な変動要因を取り出すため,市場構造の変化に柔軟に対応できる.

さらに,Fan et al. (2013) によって提案された POET (Principal Orthogonal complement Thresholding) は,PCAによって推定されたファクター構造に加え,残差行列(ファクターで説明できない部分)に対してスパース推定を行う.
具体的には,残差共分散行列の非対角要素に対して閾値処理(Thresholding)を適用し,ノイズと見なされる小さな共分散をゼロにすることで,よりロバストな推定を実現する.

\subsubsection{Ledoit-Wolf 線形縮小推定 (Linear Shrinkage)}
Ledoit and Wolf (2004)\footnote{Ledoit, O. and Wolf, M. (2004). Honey, I shrunk the sample covariance matrix. \textit{The Journal of Portfolio Management}, 30(4):110–119.} によって提案された手法であり,高次元データにおいて不安定になりがちな標本共分散行列 $\hat{\symbf{\Sigma}}$ と,構造が単純で推定誤差の少ない目標行列 $\symbf{F}$ の加重平均をとることで,推定誤差(分散)とバイアスのトレードオフを最適化する手法である.
\begin{align}
  \symbf{\Sigma}^{*} = \delta \symbf{F} + (1 - \delta) \hat{\symbf{\Sigma}}
\end{align}
ここで $\delta$ は最適な縮小強度であり,データから漸近的に最適な値が計算される.
本研究では,目標行列 $\symbf{F}$ としてスケーリングされた単位行列(すべての資産が互いに無相関で,かつ等しい分散を持つと仮定するモデル)を採用した.

\subsubsection{非線形縮小推定 (Nonlinear Shrinkage)}
線形縮小をさらに発展させた手法であり,De Nard (2022)\footnote{De Nard, G. (2022). Oops! I Shrunk the Sample Covariance Matrix Again: Blockbuster Meets Shrinkage. \textit{Journal of Financial Econometrics}, 20(4):569–611.} など近年の研究でもその有効性が議論されている.
標本共分散行列の固有値分布に対して非線形な変換(Nonlinear Transformation)を適用することで,大次元固有値問題に特有のバイアスを補正する.
線形縮小がすべての固有値を一律に(あるいは線形に)補正するのに対し,非線形縮小は個々の固有値の大きさに応じて最適な補正を行うため,より柔軟かつ高精度な推定が可能となる.

\section{実験内容}
提案手法の有効性を検証するため,世界的な株式データを用いたバックテストを行う.

\subsection{データセット}
\begin{itemize}
    \item \textbf{対象}: 世界的な株式市場における個別銘柄
    \item \textbf{期間}: 1995年3月1日から2023年12月19日
    \item \textbf{データ処理}: 日次リターンを使用し,欠損値は0で補完する.
\end{itemize}

\subsection{実験設定}
\begin{itemize}
    \item \textbf{バックテスト手法}: ローリングウィンドウ法
    \item \textbf{訓練期間 ($T_{\text{train}}$)}: 21日
    \item \textbf{ユニバースサイズ ($N$)}: 100, 200, 500 銘柄
    \item \textbf{評価指標}: 実測ボラティリティおよびシャープ・レシオ(年率換算)
\end{itemize}

\section{実験結果}

各手法を用いた最小分散ポートフォリオのパフォーマンス(平均シャープ・レシオ)を以下の表に示す.

\begin{table}[h]
\centering
\caption{各手法における平均シャープ・レシオ (Mean Sharpe Ratio)}
\begin{tabular}{lrrrrr}
\hline
Method & $N=30$ & $N=50$ & $N=100$ & $N=200$ & $N=500$ \\ \hline
Sample Covariance & -0.014 & 0.056 & -0.036 & -0.011 & 0.109 \\
Market Factor & 0.704 & 0.780 & 0.764 & 0.782 & 0.768 \\
PCA & 0.632 & 0.667 & 0.764 & 0.784 & 0.757 \\
POET & 0.535 & 0.669 & 0.766 & 0.782 & 0.754 \\
Linear Shrinkage & 0.698 & 0.790 & 0.887 & 0.947 & 1.044 \\
Nonlinear Shrinkage & 0.576 & 0.689 & 0.881 & 0.975 & 1.083 \\ \hline
\end{tabular}
\end{table}

実験の結果,以下の傾向が確認された.

\begin{enumerate}
    \item \textbf{標本共分散行列の限界}: Sample Covarianceは,いずれの $N$ においてもシャープ・レシオが非常に低く,ポートフォリオ最適化には不適であることが再確認された.
    
    \item \textbf{低次元におけるファクターモデルの有効性}: $N=30$ のように銘柄数が訓練期間 ($T=21$) に近い場合,Market Factorモデルが最も安定したパフォーマンス ($0.704$) を示した.これは,データ数が少ない状況下では,複雑な推定を行うよりも強力な事前知識(市場ファクター)を利用する方が有利であることを示唆している.
    
    \item \textbf{縮小推定の優位性とスケーラビリティ}: $N=50$ を超えると,Linear Shrinkage が Market Factor を上回り,最高のパフォーマンスを発揮するようになる.特に $N=100, 200, 500$ と次元が増加するにつれて,Linear/Nonlinear Shrinkage の優位性は顕著になる.これは,縮小推定が高次元データにおけるノイズを効果的に抑制し,かつデータの固有構造も柔軟に取り込んでいるためであると考えられる.
\end{enumerate}

\section{まとめ}
本研究では,Ledoit-Wolfの縮小推定法をはじめとする高次元共分散行列推定手法を,世界的な株式市場データに適用し評価を行った.
検証の結果,縮小推定を用いることで,従来の標本共分散行列や単純なファクターモデルを上回るポートフォリオ運用成績が得られることが確認できた.
特に,銘柄数が多く観測期間が短いという,実務的にも困難な状況下において,縮小推定はリスクを効果的に低減し,リターンを最大化するための有力なアプローチであるといえる.


\end{document}
