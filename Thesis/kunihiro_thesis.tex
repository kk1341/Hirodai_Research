%#!lualatex
\documentclass[lualatex,a4paper,11pt]{jlreq}
\usepackage{luatexja-fontspec}
\newjfontfamily{\HaraMC}[
    FontFace = {l}{n}{*-Light},
    FontFace = {m}{n}{*-Regular},
    FontFace = {b}{n}{*-Bold},
    BoldFont = {*-Bold},
  ]{HaranoAjiMincho}
\newjfontfamily{\HaraGT}[
    FontFace = {m}{n}{*-Regular},
    FontFace = {b}{n}{*-Bold},
    FontFace = {eb}{n}{*-Heavy},
    BoldFont = {*-Bold},
  ]{HaranoAjiGothic}
\newjfontfamily{\HaraMG}[
    FontFace = {m}{n}{*-Medium},
  ]{HaranoAjiGothic}

%\usepackage{luatexja-ruby}% for kenten
% [TODO] https://ja.osdn.net/projects/luatex-ja/ticket/41023
% => based on okumacro.sty
\makeatletter
\def\kenten#1{%
  \setbox1=\hbox to\z@{\hbox to 1\zw{\hss ・\hss}\hss}%
  \ht1=.63\zw
  \@kenten#1\end\relax}
\def\@kenten#1#2{%
  \ifx#1\end \let\next=\relax \else
    \raise.75\zw\copy1\nobreak #1\ifx#2\end\else
    \hskip\ltjgetparameter{kanjiskip}\relax\fi
    \let\next=\@kenten
  \fi\next#2}
\makeatother

%%%%%%%%%%%%%%%%
% additional packages
\usepackage{amsmath}
\usepackage{array}
\usepackage{bm}
% \usepackage[defaultsups]{newpxtext}
% \usepackage[zerostyle=c,straightquotes]{newtxtt}
% \usepackage{newpxmath}
\usepackage[svgnames]{xcolor}
\usepackage[hyperfootnotes=false]{hyperref}
\usepackage{hologo}
\usepackage{xspace}

% フォントエラー対策のため追加
\usepackage{fontspec}
\usepackage{unicode-math}

\setmainfont{TeX Gyre Pagella} 
\setsansfont{TeX Gyre Heros} % サンセリフの例
\setmonofont{TeX Gyre Cursor} % 等幅フォントの例
\setmathfont{STIX Two Math}

\iffalse
\def\Pkg#1{\textsf{#1}}
\def\Opt#1{\texttt{#1}}
\def\ClsY#1{\Pkg{\textcolor{Black}{#1}}}
\def\ClsT#1{\Pkg{\textcolor{Green}{#1}}}
\def\ClsYT#1{\Pkg{\textcolor{Blue}{#1}}}
\def\file#1{\textsf{#1}}
\def\code#1{\texttt{#1}}
\def\cs#1{\texttt{\textbackslash#1}}
\fi

\usepackage[japanese,english]{babel}% References in English
\AtBeginDocument{\def\figurename{Figure~}}% add a space

\usepackage{listings}
\lstset{%
  language=[LaTeX]{TeX},
  basicstyle={\small\ttfamily},
  texcsstyle={*\color{blue}},
  commentstyle={\color{Green}},
  columns=[l]{fullflexible},
  numbers=left,
  xrightmargin=0\zw,
  xleftmargin=3\zw,
  aboveskip=3pt,
  belowskip=3pt,
  moretexcs={setlength,setmainfont,maketitle,RequirePackage}
}

%\usepackage{shortvrb}
%\MakeShortVerb*{|}
%%%%%%%%%%%%%%%%

% foreign text -> italic
%\def\Foreign#1{\textit{#1}}



%def\_{\leavevmode\vrule width .45em height -.2ex depth .3ex\relax}

\frenchspacing
\begin{document}

\title{\textsf{\textbf{縮小を用いた実データにおけるポートフォリオ構築の最適化について}}}
\author{國廣 壱生
\thanks{\url{https://texjp.org},\ e-mail: \texttt{issue(at)texjp.org}}}
\maketitle


\hfil {\textbf 要約}
\vskip1em

\begin{center}
\begin{minipage}{0.8\textwidth}
  現代のポートフォリオ理論は,Markowitz(1952)によって確立された平均・分散アプローチに基づいており,所与の期待リターンに対してリスクを最小化することを目的とする.
  この最適化計算の核心には,資産リターンの標本共分散行列の推定が必要となる.
  しかし,ポートフォリオ構築時に通常用いられる標本共分散行列は,銘柄数が多い現実の状況下で推定誤差が大きいため,投資比率が不安定化する.
  この問題を解決するため,縮小という手法を用いて標本共分散行列の構造を変化させ,推定誤差を小さくする.
  これは比較的新しい手法で,実データにおける検証が不十分である.
  そのため本研究では,過去の株価データを用いて推定した共分散行列に縮小を適用し,従来よりも安定かつ効率的なポートフォリオ構築が行えるかを評価する.
\end{minipage}
\end{center}
\vskip2em

\section{導入}
本研究では,ポートフォリオ最適化問題における共分散行列の推定精度向上を目的とし,Ledoit-Wolfの縮小推定法を日本の株式市場データに適用した結果について述べる.特に,標本共分散行列が抱える推定誤差の問題と,それを縮小推定によってどのように改善できるかに焦点を当てる.

\section{関連手法}

\subsection{ポートフォリオ最適化問題}
% 研究の核である「ポートフォリオ最適化問題」に触れつつ,一番手法として古い標本共分散行列を用いた最適化手法を話す.
% ここで,標本共分散行列では推定誤差が大きいという問題があると指摘して,新しい手法が必要な理由を説明

\subsubsection{最小分散ポートフォリオ}

現代ポートフォリオ理論は,Harry Markowitzが1952年に発表した論文で提唱した最適化理論を元に構築されており,投資家が直面するリターンとリスクのトレードオフを定量的に解決する方法に従う.
この理論の目的は特定のリスク水準の下で期待リターンを最大化し,特定の期待リターン水準を達成する中でリスクを最小化することにある.

ポートフォリオの期待リターン $\mathbb{E}(r_p)$ は,個々の資産の期待リターン $\mathbb{E}(r_i)$ を,その資産の投資ウェイト $w_i$ で加重平均した値で求める.

\begin{align}
    \mathbb{E}(r_p) = \sum_{i=1}^{N} w_i \mathbb{E}(r_i) = \symbf{w}^\top \symbf{\mu}
\end{align}

$\symbf{w}$ は各資産に対する投資比率を縦に並べたウェイトベクトルであり, $\symbf{\mu}$ は各資産の期待リターンを並べた列ベクトルである.
また,ポートフォリオの分散 $\sigma^2_p$ は,ポートフォリオのリスクの大きさを表す.単に各資産の分散を合計するのではなく,資産間の共分散を計算することで求める.

\begin{align}
  \sigma^2_p = \sum_{i=1}^{N} \sum_{j=1}^{N} w_i w_j \text{Cov}(r_i, r_j) = \symbf{w}^\top \symbf{\Sigma} \symbf{w}
\end{align}

$\text{Cov}(r_i, r_j)$ は資産 $i$ のリターン $r_i$ と資産 $j$ のリターン $r_j$ の共分散でリターンの相関を示し, $\symbf{\Sigma}$ はすべての資産ペアの共分散と分散をまとめた分散共分散行列である.
この分散の計算において,資産間の共分散が小さい (または負の相関がある) ほど,ポートフォリオ全体のリスクは小さくなる.
そのため,ポートフォリオを最適化するには,分散共分散行列である $\symbf{\Sigma}$ を最小化する必要がある.

したがって,最小分散ポートフォリオを求める最小化問題は以下のように定式化できる.
\begin{align}
  \min_{\symbf{w}} \quad \symbf{w}^\top \symbf{\Sigma} \symbf{w}
  \\[1em]
  \text{s.t.} \quad \symbf{C}^\top\symbf{w} = \symbf{\gamma}
\end{align}

$\symbf{C}$ は, $N$ を資産数, $M$ を制約数としたときにおける $N \times M$ の行列であり,$\symbf{\gamma}$ は各制約条件の目標値を格納した $M \times 1$ の列ベクトルである.
この最小化問題を解くことによって,リスクを最小限に抑える最適なウェイトベクトル $\symbf{w}$ が以下のように求められることが知られている.
\begin{align}
  \symbf{w} &= \symbf{\Sigma}^{-1} \symbf{C} (\symbf{C}^\top \symbf{\Sigma}^{-1} \symbf{C})^{-1} \symbf{\gamma} \\ \notag
             &= \sum_{i=1}^{N} \frac{\symbf{C}^\top \cdot \symbf{u}_i}{\lambda_i} \symbf{u}_i
\end{align}
ここで,$\lambda_i$は分散共分散行列の各固有値を表し,$\symbf{u}_i$は分散共分散行列の各固有ベクトルを表す.

\subsubsection{標本共分散行列}

理論的なポートフォリオ最小化問題では,真の分散共分散行列 $\symbf{\Sigma}$ を用いて推定されるが,実務上では $\symbf{\Sigma}$ を観測することは不可能であるため,
過去の市場データから推定した標本共分散行列 $\hat{\symbf{\Sigma}}$ が用いられる.

実務上では,ポートフォリオの最適化は(3)式を用いて以下のように行われる.
\begin{align}
  \min_{\symbf{w}} \quad \symbf{w}^\top \hat{\symbf{\Sigma}} \symbf{w}
  \\[1em]
  \text{s.t.} \quad \symbf{C}^\top\symbf{w} = \symbf{\gamma}
\end{align}

また,(5)式も標本共分散行列を用いて以下のように表される.
\begin{align}
  \symbf{w} &= \hat{\symbf{\Sigma}}^{-1} \symbf{C} (\symbf{C}^\top \hat{\symbf{\Sigma}}^{-1} \symbf{C})^{-1} \symbf{\gamma} \\ \notag
             &= \sum_{i=1}^{N} \frac{\symbf{C}^\top \cdot \symbf{u}_i}{\lambda_i} \symbf{u}_i
\end{align}

しかし,銘柄数 $N$ が観測数 $T$ よりも多い場合では,分散共分散行列の推定に必要な観測数 $T$ が少ないため,取得できるデータが希薄になり標本誤差が大きくなってしまう.
このとき,標本共分散行列のランクは $T$ 以下となり,本来の行列の次元 $N$ よりも小さくなってしまうため逆行列を持たなくなる.
また,銘柄数 $N$ が観測数 $T$ に近い場合でも,標本共分散行列が持つ最小固有値の値が限りなくゼロの方向へ歪んで推定される.そのため,ポートフォリオの最適なウェイトベクトルを計算する際にほぼ無限大のウェイトが割り当てられてしまう.
これは,ポートフォリオの真のリスクを著しく過小評価することになり,正しくポートフォリオの最適なウェイトベクトルが推定できない.
したがって,標本共分散行列をポートフォリオ最適化問題に使うことは不適切であり,他の手法を用いてポートフォリオ最適化問題を解く必要がある.

\subsection{縮小}

\subsection{縮小推定量の形式と目標行列}
ポートフォリオ最適化問題では,銘柄数 $N$ が観測数 $T$ よりも多い場合では逆行列が推定できなくなり,銘柄数 $N$ が観測数 $T$ が同程度の場合では,標本共分散行列の固有値が過度に分散してしまう問題があった.
特に最小固有値はゼロに偏り,その逆行列を用いたポートフォリオウェイトの計算が極端に不安定になる.
そこで,標本共分散行列 $\hat{\symbf{\Sigma}}$ の情報 ( $N \times N$ の各要素 ) を,より単純で構造化された目標行列 $F$ に向かって変化させることを考える.
この方法を「縮小」と言い,これによって改善される共分散行列の推定量 $\symbf{\Sigma}^{*}$ は,以下の加重平均として定義される.

\begin{align}
  \symbf{\Sigma}^{*} = \delta \symbf{F} + (1 - \delta) \hat{\symbf{\Sigma}}
\end{align}

$\delta \in [0, 1]$ は縮小パラメータであり,最適な推定を行うためのパラメータである.

また,目標行列 $\symbf{F}$ は,真の共分散行列 $\symbf{\Sigma}$ の特性を最もよく捉えていると考えられる,構造化された単純な行列が選ばれる.
目標行列の主な例としては以下のいずれかである.
\begin{itemize}
  \item 単一指標モデル (Single-Index Model) に基づく行列
    \begin{itemize}
      \item 全銘柄の分散と共分散を,単一のファクター (例:市場インデックス) のみで説明できると仮定する行列
    \end{itemize}
  \item 対角行列
    \begin{itemize}
      \item 全銘柄の共分散はゼロ (無相関) であると仮定し,分散情報のみを保持する単純なモデル
    \end{itemize}
  \item すべての相関係数が等しい行列
  \begin{itemize}
    \item 全銘柄の相関係数が一定値であると仮定する行列  
  \end{itemize}
\end{itemize}

これらの行列は常に逆行列を持つことが保証されており,その構造が単純な分,推定誤差が非常に小さいというメリットがある.

\subsection{縮小パラメータ}
本研究では,Ledoit and Wolf (2004) が提案した手法を用い,漸近的に最適な縮小強度 $\delta$ をデータから推定する.
目標行列 $\symbf{F}$ としては,スケーリングされた単位行列(定数分散モデルに相当)を採用する.
具体的には,$\symbf{\mu} = \text{Tr}(\hat{\symbf{\Sigma}}) / N$ とし,$\symbf{F} = \mu \symbf{I}$ とする.

最適な縮小強度は,期待損失 $E[\| \symbf{\Sigma}^* - \symbf{\Sigma} \|^2_F]$ を最小化するように決定される.
これにより,推定誤差が大きい場合は $\delta$ が大きくなり(目標行列に近づき),逆に推定精度が高い場合は $\delta$ が小さくなる(標本共分散行列を尊重する).
本実験におけるプログラムでも,各タイムステップごとにこの最適な $\delta$ を再計算し,動的に共分散行列を更新している.

\section{実験内容}
提案手法の有効性を検証するため,日本の株式データを用いたバックテストを行う.

\subsection{データセット}
\begin{itemize}
    \item \textbf{対象}: 日本の株式市場における個別銘柄(CSVファイルとして提供されたテキストデータ)
    \item \textbf{期間}: 1995年3月1日から2023年12月19日
    \item \textbf{データ処理}: 
    \begin{itemize}
        \item 各銘柄の日次リターン(RETX)を使用する.
        \item 欠損値については,取引がない日やデータ不足の場合を考慮し,0で補完する(ゼロ埋め).これは保守的な評価を行うためである.
    \end{itemize}
\end{itemize}

\subsection{実験設定}
\begin{itemize}
    \item \textbf{バックテスト手法}: ローリングウィンドウ法
    \item \textbf{訓練期間 ($T_{\text{train}}$)}: 21日(約1ヶ月の営業日)
    \item \textbf{評価指標}:
    \begin{itemize}
        \item \textbf{実測ボラティリティ}: ポートフォリオのリターンの標準偏差(リスク).低いほど安定していることを示す.
        \item \textbf{シャープ・レシオ}: リスクフリーレートを0とした場合のリスク調整後リターン(平均リターン / 標準偏差).高いほど運用効率が良い.
    \end{itemize}
\end{itemize}

\section{実験結果}
実験では,通常の標本共分散行列(Sample Covariance)を用いた場合と,縮小推定(Ledoit-Wolf Shrinkage)を用いた場合の最小分散ポートフォリオのパフォーマンスを比較した.

実験の結果,以下の傾向が確認された.
\begin{enumerate}
    \item \textbf{リスクの低減}: 縮小推定を用いたポートフォリオは,標本共分散行列を用いた場合に比べて実測ボラティリティが低下する傾向が見られた.これは,縮小によって極端な相関の推定誤差が抑制され,より安定した分散共分散行列が得られたためであると考えられる.
    \item \textbf{運用効率の改善}: シャープ・レシオにおいても,縮小推定の方が高い値を示すケースが多く,リスクあたりのリターンが改善した.
    \item \textbf{極端なウェイトの抑制}: 標本共分散行列では,特定の銘柄に極端なショート(空売り)やロング(買い)のポジションが集中しやすいが,縮小推定を導入することでウェイトが平準化され,現実的なポートフォリオ構築が可能となった.
\end{enumerate}

特に,銘柄数 $N$ が訓練期間 $T_{\text{train}}$ に比べて大きい場合($N > T$),標本共分散行列はランク落ちなどにより逆行列が計算不能,あるいは非常に不安定になるが,縮小推定では常に正則であることが保証されるため,安定して計算を行うことができた.

\section{まとめ}
本研究では,ポートフォリオ最適化における標本共分散行列の推定誤差問題に対処するため,Ledoit-Wolfの縮小推定法を適用し,その有効性を実データを用いて検証した.
結果として,縮小推定を用いることでポートフォリオのリスク(ボラティリティ)を低減し,シャープ・レシオを向上させることが確認できた.
特に,データの期間が短い場合や銘柄数が多い場合において,縮小推定は非常に強力なツールとなる.
今後は,異なる目標行列(例えば,市場ファクターモデル)を用いた場合の比較や,より長期的な期間での安定性の検証が課題である.


\end{document}
